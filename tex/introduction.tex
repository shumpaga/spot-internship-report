\chapter{Introduction}

This internship took place at \LRDE\space (Research and Development Laboratory of \EPITA). It was conducted
under the supervisation of \textit{Alexandre Duret-Lutz}, assistant professor, both teacher at \EPITA\space
and researcher in the laboratory.\\

Thanks to my father, \textit{Gérard Gbaguidi Aïsse}, senior lecturer in civil engineering, I have always
been interested in research. I chose to apply for an internship at \LRDE\space to confirm or disprove this 
attraction towards the world of research.\\

The thought of having an experience in this laboratory came naturally at the end of the first
engineering year. This year covers various domains of computer engineering through many programming
assignments and courses. It aims to bring every students to
discover those various aspects of computer engineering and identify their favourite field of work. During
this year, one of the facts that came out the most is that I like to be confronted with problems of
mathematical and algorithmic character. This lead me to realize that for my last two years at \EPITA\space
I want to specialize in cognitive science and artificial intelligence (required course when working as
student in the laboratory). All this reinforced my desire of applying to \LRDE.\\

Among the laboratory's projects I chose to ask for an internship on Spot. This is for three
main reasons. The first one is that it exploits \textit{Automata Theory}, an interesting course we
have been introduced to, and \textit{Model Checking} that aroused my curiosity. The
second reason is that this project is mainly written in C++, a language that I like for its
expressional strength which evolves quickly at each revision. And the last one is that
\textit{Alexandre Duret-Lutz}, the supervisor of this project is one of my favourite teacher at \EPITA\space
and he really makes me want to study more of his courses.\\

Spot is a C++ library for Linear-time Temporal Logic (LTL), $\omega$-automata manipulation and model
checking. In addition to the C++ interface it provides some command-line tools, a Python binding and an
on-line translator.\\

My entire internship was focused on improving SAT-based minimization of $\omega$-automata. This minimization
is the result of two publications, an initial one realized in 2014 and an improvement made in 2015.
This subject was already a topic of concern as minimizing $\omega$-automata usually makes algorithms that
work on it more efficient.\\

Until now, SAT-based minimization procedures were using an external SAT solver. The main objective of this
internship was to choose a SAT solver, integrate it to Spot and implement different approaches of
minimization (notably incremental SAT solving) in order to find out the most efficient.\\

This internship report will start with an introduction of the \LRDE\space, its projects including Spot and the
basics concepts related to. Then, it will define the outline and objectives of this intership before
explaining the completed work.
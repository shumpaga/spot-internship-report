\chapter{Conclusion}
This internship of 5 months (from the 29th of August 2016 to the 15th of January 2017) was a very
fascinating experience. As the last internship of the common core program before our specialization, its
objective is to help us choosing the best academic path for the next phases.\\

For me, these 5 months of working have met my expectations. Working in a scientific laboratory,
being in contact with computer researchers, learning from them, understanding how a laboratory works,
attending seminars organized by the laboratory and taught by some external professionals, in short, being
in such an environment helped me picturing with more precision how would be my future experiences as
a research student.\\

It was an enriching internship as it taught me a lot. Indeed, this enabled me to understand the slight
errors I was used to making while programming in C++, to learn from the existing efficient and clean code,
to get valuable feedback from my supervisor, to have a better handling of Python and Shell programming.
I also learned about some languages as R, perl that was new to me. Working in such a big project made me
master some good behaviours while coding, especially how to make a better use of Git, which is really
important when numerous people are working on a specific project at the same time. It also helped me to be
more rigorous, consistant in my work, concerned about the user experience, the clarity of the code, the strength of
the tests.\\

Obviously it also taught me a lot about the $\omega$-automata and theoritical concepts used in Spot.
Before this internship, I never heard about satisfiability problem, the existence of SAT solvers. Now,
even if I do not know how they are implemented, I understand in which cases they can be really useful and
how to use them.\\

Reciprocally, Spot SAT-based minimization has been improved. It has acquired some new algorithms, which
have stirred up some new questions or axis of research. For instance, why incremental SAT solving did not
meet all our expectations? However, Spot now distributes its own SAT solver, PicoSAT. Its SAT-based
minimization is definitely faster than before and less memory-hungry.\\

This internship was a really positive experience that was concluded by my acceptance at the laboratory
as a CSI student. I will be continuing to contribute on the Spot project and we have already discussed about my
next subject, we want Spot to support some new algorithms concerning co-Büchi automata that have been
described in some papers.
